%!TEX root = ecosystem.tex

\rSec0[intro.defs]{Terms and definitions}

\pnum
\indextext{definitions|(}%
For the purposes of this document,
the terms and definitions given in ISO/IEC 2382,
the terms and definitions given in ISO/IEC 14882:2020,
and the following apply.

\pnum
ISO and IEC maintain terminology databases for use in standardization
at the following addresses:
\begin{itemize}
\item ISO Online browsing platform: available at \url{https://www.iso.org/obp}
\item IEC Electropedia: available at \url{https://www.electropedia.org/}
\end{itemize}

\pnum
Terms that are used only in a small portion of this document are defined where
they are used and italicized where they are defined.

\indexdefn{application}%
\definition{application}{defns.application}
a computer program that performs some desired function.

\begin{defnote}
From POSIX.
\end{defnote}

\indexdefn{capability}%
\definition{capability}{defns.capability}
an aspect of an overall specification that defines a subset of the entire
specification.

\indexdefn{directory}%
\definition{directory}{defns.directory}
a file that contains directory entries.

\begin{defnote}
From POSIX.
\end{defnote}

\indexdefn{directory entry}%
\definition{directory entry}{defns.direntry}
an object that associates a filename with a file.

\begin{defnote}
From POSIX.
\end{defnote}

\indexdefn{file}%
\definition{file}{defns.file}
an object that can be written to, or read from, or both.

\begin{defnote}
From POSIX.
\end{defnote}

\indexdefn{filename}%
\definition{filename}{defns.filename}
a sequence of bytes used to name a file.

\begin{defnote}
From POSIX.
\end{defnote}

\indexdefn{parent directory}%
\definition{parent directory}{defns.parentdir}
a directory containing a directory entry for the file under discussion.

\begin{defnote}
From POSIX.
\end{defnote}

\indexdefn{pathname}%
\definition{pathname}{defns.pathname}
a string that is used to identify a file.

\begin{defnote}
From POSIX.
\end{defnote}
