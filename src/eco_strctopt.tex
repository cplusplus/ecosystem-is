%!TEX root = ecosystem.tex

% strctopt
\rSec0[strctopt]{Structured Options}


% strctopt.pre
\rSec1[strctopt.pre]{Preamble}

\pnum
This clause describes options, output, and formats that control the behavior of
applications through the specification of arguments and options from
\emph{structured options}.

\pnum
This clause specifies the \verb|std.strctopt| capability \iref{intspct.cap}
version \verb|1.0.0|.

\pnum
An application can implement this capability.

\pnum
An application that implements the \verb|std.strctopt| capability shall include
the \verb|std.strctopt| field and version value in the introspection JSON text
output \iref{intspct.schema.cap}.

% strctopt.overview
\rSec1[strctopt.overview]{Overview}

\pnum
\begin{outputblock}
@\emph{application}@ [ @\grammarterm{std-strctopt-input-opt}@ [@\emph{file}@] ]
\end{outputblock}

\rSec1[strctopt.input]{Input Option}

\pnum
\grammarterm{std-strctopt-input-opt}
\begin{indented}
	The pathname of a file to read the \emph{structured options} from. The
	option is specified as \verb@--std-opt=file@ or \verb@-std-opt:file@. If
	\emph{file} is ‘\verb@-@’, the standard input shall be used.
\end{indented}

\rSec1[strctopt.file]{Files}

\pnum
An application shall read a valid JSON text file that conforms to the
\emph{structured options} schema \iref{strctopt.schema}.

\pnum
An application shall interpret the information in the file as if the options
and arguments in the file occur in the same position as the
\grammarterm{std-strctopt-input-opt} argument of the list of arguments given to
the application. Given either directly as part of the application command line
or as part of the arguments field \iref{strctopt.schema.args}.

\rSec1[strctopt.schema]{Schema}

\pnum
A \emph{structured options} JSON text file shall contain one \emph{structured
options} JSON object \iref{strctopt.schema.obj}.

\rSec2[strctopt.schema.obj]{Structured Options Object}

\pnum
The \emph{structured options} object is the root JSON object of the structured
options JSON text.

\pnum
A \emph{structure options} object can have the following fields.

\rSec2[strctopt.schema.schema]{JSON Schema Field}

\begin{itemdescr}

	\pnum \fldname \verb|$schema|

	\pnum \fldtype \verb|string|

	\pnum \fldval
	The value shall be a reference to a JSON Schema specification.

	\pnum \flddesc
	A \emph{structure options} object can contain this field. If a
	\emph{structure options} object does not contain this field the value shall
	be a reference to the JSON Schema corresponding to the current edition of
	this standard (strctoptjschm).

\end{itemdescr}

\rSec2[strctopt.schema.ver]{Version Field}

\begin{itemdescr}

	\pnum \fldname \verb|version|

	\pnum \fldtype \verb|string|

	\pnum \fldval
	\verb|1| or \verb|1.0| or \verb|1.0.0|.

	\pnum
	\flddesc
	The version field indicates the version of the \emph{structured options}
	represented in the contents of the JSON text.

\end{itemdescr}

\rSec2[strctopt.schema.args]{Arguments Field}

\begin{itemdescr}

	\pnum \fldname \verb|arguments|

	\pnum \fldtype \verb|array|

	\pnum \fldval
	The value shall be a JSON \verb|array|. The items in the \verb|array| shall
	be of JSON \verb|string| types.

	\pnum \flddesc
	The arguments field specifies items to be interpreted directly as if they
	occur in the command line of the program.

\end{itemdescr}

The application shall process the items as if they replace the
\grammarterm{std-strctopt-input-opt} argument.

\rSec2[strctopt.schema.opts]{Options Field}

\begin{itemdescr}

	\pnum \fldname \verb|options|

	\pnum \fldtype \verb|array|

	\pnum \fldval
	The value shall be a JSON \verb|array|. The items in the \verb|array| shall
	be of JSON \verb|string| or \verb|object| types.

	\pnum \flditems
	(for \verb|string|) The item shall be a single \emph{flag option}.

	\pnum \flditems
	(for \verb|array|) The item shall be a single \emph{structured option}.

	\pnum \flddesc
	A \emph{structured options} object can contain this field.

\end{itemdescr}

\rSec3[strctopt.schema.names]{Names}

\pnum
\begin{ncbnf}

\nontermdef{name}\br
	\descr{one or more of:}\br
	\ucode{0061} .. \ucode{007A} \uname{LATIN SMALL LETTER A .. Z}\br
	\ucode{0030} .. \ucode{0039} \uname{DIGIT ZERO .. NINE}\br
	\ucode{005F} \uname{LOW LINE} \ucode{002D} \uname{HYPHEN-MINUS}

\nontermdef{scope}\br
	name \opt{scope-designator}

\nontermdef{scope-designator}\br
	\ucode{002E} \uname{FULL STOP}

\end{ncbnf}

\pnum
The \grammarterm{name} \verb|std| is reserved for \emph{flag options} and
\emph{structured options} defined in this standard.

\pnum
Applications can specify vendor designated \grammarterm{name} parts outside of
this standard.

\rSec2[strctopt.schema.flag]{Flag Option}

\pnum
\begin{ncbnf}

\nontermdef{flag-option}\br
	\descr{one of:} yes-flag no-flag

\nontermdef{yes-flag}\br
	\descr{one of:} yes-flag-name scoped-yes-flag-name

\nontermdef{yes-flag-name}\br
	name

\nontermdef{scoped-yes-flag-name}\br
	\opt{scope} name

\nontermdef{no-flag}\br
	\descr{one of:} no-flag-name scoped-no-flag-name

\nontermdef{no-flag-name}\br
	no-flag-prefix name

\nontermdef{scoped-no-flag-name}\br
	\opt{scope} no-flag-name

\nontermdef{no-flag-prefix}\br
	\ucode{006E} \uname{LATIN SMALL LETTER N}
	\ucode{006F} \uname{LATIN SMALL LETTER O}
	\ucode{002D} \uname{HYPHEN-MINUS}

\end{ncbnf}

\pnum
A \emph{flag option} is an option that allows or prevents the specified
semantics of the option.

\pnum
A \emph{flag option} shall be either a \grammarterm{yes-flag} or
\grammarterm{no-flag}.

\pnum
A \grammarterm{yes-flag} allows the semantics in the application as specified
by effects of that behavior.

\pnum
A \grammarterm{no-flag} prevents the semantics in the application as specified
by effects of that behavior.

\pnum
A \emph{flag option} without a \grammarterm{scope} shall be equivalent the same
\emph{flag option} with a \verb|std.| scope.

\rSec2[strctopt.schema.struct]{Structured Option}

\pnum
A \emph{structured option} JSON object shall contain a
\grammarterm{structured-option-name} field.

\pnum
A \emph{structured option} JSON object can contain additional fields as
specified by the option specification.

\begin{itemdescr}

	\pnum \fldname \verb|name|

	\pnum \fldtype \verb|string|

	\pnum \fldval
	A valid \grammarterm{structured-option-name}.

	\pnum \flddesc
	The name of the \emph{structured options}.

\end{itemdescr}

\pnum
\begin{ncbnf}

\nontermdef{structured-option-name}\br
	\opt{scope} name

\end{ncbnf}

\pnum
A \emph{structured option} without a \grammarterm{scope} shall be equivalent to
the same \emph{structured option} with a \verb|std.| \grammarterm{scope}.

\rSec2[strctopt.schema.opt.opt]{Option std.opt}

\pnum
The \verb|std.opt| option defines an option to refer to additional structured
options \iref{strctopt} to process.

\pnum
An application shall implement this option.

\pnum
The \verb|std.opt| option shall have the following fields.

\begin{itemdescr}

	\pnum \fldname \verb|name|

	\pnum \fldtype \verb|string|

	\pnum \fldval \verb|std.opt|

	\pnum \flddesc
	The name of the option.

\end{itemdescr}

\begin{itemdescr}

	\pnum \fldname \verb|files|

	\pnum \fldtype \verb|string| or \verb|array|

	\pnum \fldval (for \verb|string|)
	A pathname to a file containing structured options \iref{strctopt}.

	\pnum \fldval (for \verb|array|)
	A list of pathname \verb|string| items to files containing structured
	options \iref{strctopt}.

	\pnum \flddesc
	One or more references to files that include additional structured options
	\iref{strctopt} that are processed as if they occur at the location of the
	\verb|std.opt| option.

\end{itemdescr}
