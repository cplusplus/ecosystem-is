%!TEX root = ecosystem.tex

% strctparam
\rSec0[strctparam]{Structured Parameters}


% strctparam.pre
\rSec1[strctparam.pre]{Preamble}

\pnum
This clause describes options, output, and formats that control the behavior of
applications through the specification of arguments and options from
\emph{structured parameters}.

\pnum
This clause specifies the \verb|std.strctparam| capability \iref{intspct.cap}
version \verb|1.0.0|.

\pnum
An application can implement this capability.

\pnum
An application that implements the \verb|std.strctparam| capability shall
include the \verb|std.strctparam| field and version value in the introspection
JSON text output \iref{intspct.schema.cap}.

% strctparam.overview
\rSec1[strctparam.overview]{Overview}

\pnum
\begin{outputblock}
@\emph{application}@ [ @\grammarterm{std-strctparam-input}@ [@\emph{file}@] ]
\end{outputblock}

\rSec1[strctparam.input]{Input Option}

\pnum
\grammarterm{std-strctparam-input}
\begin{indented}
	The pathname of a file to read the \emph{structured parameters} from. The
	option is specified as \verb@--std-param=file@ or \verb@-std-param:file@. If
	\emph{file} is ‘\verb@-@’, the standard input shall be used.
\end{indented}

\rSec1[strctparam.file]{Files}

\pnum
An application shall read a valid JSON text file that conforms to the
\emph{structured parameters} schema \iref{strctparam.schema}.

\pnum
An application shall interpret the information in the file as if the options
and arguments in the file occur in the same position as the
\grammarterm{std-strctparam-input} parameter of the list of parameters given to
the application. Given either directly as part of the application command line
or as part of the arguments field \iref{strctparam.schema.args}.

\pnum
An application shall \emph{process} an argument or option in the file by
applying the semantics to change the state of the application that is relevant 
to the argument or option at the point that the application sees the argument 
or option.

\rSec1[strctparam.schema]{Schema}

\pnum
A \emph{structured parameters} JSON text file shall contain one \emph{structured
parameters} JSON object \iref{strctparam.schema.obj}.

\rSec2[strctparam.schema.obj]{Structured Parameters Object}

\pnum
The \emph{structured parameters} object is the root JSON object of the 
structured parameters JSON text.

\pnum
A \emph{structured parameters} object can have the following fields.

\pnum
A \emph{structured parameters} object shall have only one of the 
\verb|arguments| and \verb|options| fields.

\rSec2[strctparam.schema.schema]{JSON Schema Field}

\begin{itemdescr}

	\pnum \fldname \verb|$schema|

	\pnum \fldtype \verb|string|

	\pnum \fldval
	The value shall be a reference to a JSON Schema specification.

	\pnum \flddesc
	A \emph{structured parameters} object can contain this field. If a
	\emph{structured parameters} object does not contain this field the value 
	shall be a reference to the JSON Schema corresponding to the current 
	edition of this standard (strctparamjschm).

\end{itemdescr}

\rSec2[strctparam.schema.ver]{Version Field}

\begin{itemdescr}

	\pnum \fldname \verb|version|

	\pnum \fldtype \verb|string|

	\pnum \fldval
	\verb|1| or \verb|1.0| or \verb|1.0.0|.

	\pnum
	\flddesc
	The version field indicates the version of the \emph{structured parameters}
	represented in the contents of the JSON text. If a \emph{Structured
	parameters} object does not contain this field the value shall be
	\verb|1.0.0|.

\end{itemdescr}

\rSec2[strctparam.schema.args]{Arguments Field}

\begin{itemdescr}

	\pnum \fldname \verb|arguments|

	\pnum \fldtype \verb|array|

	\pnum \fldval
	The value shall be a JSON \verb|array|. The items in the \verb|array| shall
	be of JSON \verb|string| types.

	\pnum \flddesc
	The arguments field specifies items to be interpreted directly as if they
	occur in the command line of the program.

\end{itemdescr}

The application shall process the items as if they replace the
\grammarterm{std-strctparam-input} argument.

\rSec2[strctparam.schema.opts]{Options Field}

\begin{itemdescr}

	\pnum \fldname \verb|options|

	\pnum \fldtype \verb|array|

	\pnum \fldval
	The value shall be a JSON \verb|array|. The items in the \verb|array| shall
	be of JSON \verb|string| or \verb|object| types.

	\pnum \flditems
	(for \verb|string|) The item shall be a single \emph{flag option}.

	\pnum \flditems
	(for \verb|array|) The item shall be a single \emph{structured option}.

	\pnum \flddesc
	A \emph{structured parameters} object can contain this field.

\end{itemdescr}

\rSec3[strctparam.schema.names]{Names}

\pnum
\begin{ncbnf}

\nontermdef{name}\br
	\descr{one or more of:}\br
	\ucode{0061} .. \ucode{007A} \uname{LATIN SMALL LETTER A .. Z}\br
	\ucode{0030} .. \ucode{0039} \uname{DIGIT ZERO .. NINE}\br
	\ucode{005F} \uname{LOW LINE} \ucode{002D} \uname{HYPHEN-MINUS}

\nontermdef{scope}\br
	name \opt{scope-designator}

\nontermdef{scope-designator}\br
	\ucode{002E} \uname{FULL STOP}

\end{ncbnf}

\pnum
The \grammarterm{name} \verb|std| is reserved for \emph{flag options} and
\emph{structured options} defined in this standard.

\pnum
Applications can specify vendor designated \grammarterm{name} parts outside of
this standard.

\rSec2[strctparam.schema.flag]{Flag Option}

\pnum
\begin{ncbnf}

\nontermdef{flag-option}\br
	\descr{one of:} yes-flag no-flag

\nontermdef{yes-flag}\br
	\descr{one of:} yes-flag-name scoped-yes-flag-name

\nontermdef{yes-flag-name}\br
	name

\nontermdef{scoped-yes-flag-name}\br
	\opt{scope} name

\nontermdef{no-flag}\br
	\descr{one of:} no-flag-name scoped-no-flag-name

\nontermdef{no-flag-name}\br
	no-flag-prefix name

\nontermdef{scoped-no-flag-name}\br
	\opt{scope} no-flag-name

\nontermdef{no-flag-prefix}\br
	\ucode{006E} \uname{LATIN SMALL LETTER N}
	\ucode{006F} \uname{LATIN SMALL LETTER O}
	\ucode{002D} \uname{HYPHEN-MINUS}

\end{ncbnf}

\pnum
A \emph{flag option} is an option that allows or prevents the specified
semantics of the option.

\pnum
A \emph{flag option} shall be either a \grammarterm{yes-flag} or
\grammarterm{no-flag}.

\pnum
A \grammarterm{yes-flag} allows the semantics in the application as specified
by effects of that behavior.

\pnum
A \grammarterm{no-flag} prevents the semantics in the application as specified
by effects of that behavior.

\pnum
A \emph{flag option} without a \grammarterm{scope} shall be equivalent the same
\emph{flag option} with a \verb|std.| scope.

\rSec2[strctparam.schema.struct]{Structured Option}

\pnum
A \emph{structured option} JSON object shall contain a
\grammarterm{structured-option-name} field.

\pnum
A \emph{structured option} JSON object can contain additional fields as
specified by the option specification.

\begin{itemdescr}

	\pnum \fldname \verb|name|

	\pnum \fldtype \verb|string|

	\pnum \fldval
	A valid \grammarterm{structured-option-name}.

	\pnum \flddesc
	The name of the \emph{structured parameters}.

\end{itemdescr}

\pnum
\begin{ncbnf}

\nontermdef{structured-option-name}\br
	\opt{scope} name

\end{ncbnf}

\pnum
A \emph{structured option} without a \grammarterm{scope} shall be equivalent to
the same \emph{structured option} with a \verb|std.| \grammarterm{scope}.

\rSec2[strctparam.schema.opt.param]{Option std.param}

\pnum
The \verb|std.param| option defines an option to refer to additional structured
options \iref{strctparam} to process.

\pnum
An application shall implement this option.

\pnum
The \verb|std.param| option shall have the following fields.

\vspace{\baselineskip}

\begin{itemdescr}

	\pnum \fldname \verb|name|

	\pnum \fldtype \verb|string|

	\pnum \fldval \verb|std.param|

	\pnum \flddesc
	The name of the option.

\end{itemdescr}

\vspace{\baselineskip}

\begin{itemdescr}

	\pnum \fldname \verb|files|

	\pnum \fldtype \verb|string| or \verb|array|

	\pnum \fldval (for \verb|string|)
	A pathname to a file containing \emph{structured parameters}
	\iref{strctparam}.

	\pnum \fldval (for \verb|array|)
	A list of pathname \verb|string| items to files containing \emph{structured 
	parameters} \iref{strctparam}.

	\pnum \flddesc
	One or more references to files that include additional \emph{structured 
	parameters} \iref{strctparam} that are processed as if they occur at the 
	location of the \verb|std.param| option.

\end{itemdescr}
