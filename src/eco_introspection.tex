%!TEX root = ecosystem.tex

% Generated file. Do not edit.

% Copyright 2024 René Ferdinand Rivera Morell
% Creative Commons Attribution 4.0 International License (CC BY 4.0)

% Introspection
\rSec0[intspct]{Introspection}

% Preamble
\rSec1[intspct.pre]{Preamble}

\pnum
This clause describes options, output, and formats that describe what
capabilities of this standard an application supports. An application shall
support the \emph{minimum level} functionality (\ref{intspct.min}). An
application can support the \emph{full level} functionality (\ref{intspct.full}).

\pnum
This clause specifies the \verb|std.info| capability (\ref{intspct.cap}).

% Overview
\rSec1[intspct.overview]{Overview}

\pnum
\begin{lstlisting}{}
@\emph{application}@ [ @\grammarterm{std-info-opt}@ [ @\grammarterm{declaration}@ ] ] [ @\grammarterm{std-info-out-opt}@ @\emph{file}@ ]
\end{lstlisting}

% Options
\rSec1[intspct.options]{Options}

\pnum
Applications shall accept one of two options syntax variations:
\verb|--name=value| (\verb|--name| without a value) or
\verb|-name:value| (\verb|-name| without a value).

\pnum
Applications shall indicate an error if invoked with an option syntax variation
that it does not support.

\begin{note}
An application will report the error in what is conventional for the
	platform it runs in. On POSIX and Windows it would return an error code,
	and optionally output to the error stream.
\end{note}

\begin{note}
It is up to a program that interacts with an application implementing
	introspection to determine what option syntax variation the application
	supports. One method to accomplish that is to execute the application with
	one of the two syntax styles and use the error indication to conclude which
	syntax works. Another is to have a priori knowledge of which syntax
	variation works.
\end{note}

% Information Option
\rSec2[intspct.opt.info]{Information Option}

\pnum
This option shall be supported.

\pnum
\grammarterm{std-info-opt}

\begin{indented}
\pnum
Outputs the version information of the capabilities supported by the
application.
The option is specified as \verb|--std-info| or \verb|-std-info|.
The option can be specified zero or one time.
The application shall support the option for \emph{minimum level}
(\ref{intspct.min}) functionality.
\end{indented}

% Information Output Option
\rSec2[intspct.opt.out]{Information Output Option}

\pnum
This option shall be supported.

\pnum
\grammarterm{std-info-out-opt} \emph{file}

\begin{indented}
\pnum
The pathname of a file to output the information to.
The option is specified as \verb|--std-info-out=file| or
\verb|-std-info-out:file|.
If \emph{file} is ‘\verb|-|’, the standard output shall be used.
The application shall support the option for \emph{minimum level}
(\ref{intspct.min}) functionality.
Not specifying this option while specifying the
\grammarterm{std-info-opt} option (\ref{intspct.opt.info}) shall be
equivalent to also specifying a \grammarterm{std-info-out-opt}
\grammarterm{file} option where \emph{file} is ‘\verb|-|’.
\end{indented}

% Declaration Option
\rSec2[intspct.opt.decl]{Declaration Option}

\pnum
This option should be supported.

\pnum
\grammarterm{std-info-opt} \emph{declaration}

\begin{indented}
\pnum
Declares the required capability version of the application.
The option is specified as \verb|--std-info=|\emph{declaration} or
\verb|-std-info:|\emph{declaration}.
The option can be specified any number of times.
The application shall support the option for \emph{full level}
(\ref{intspct.full}) functionality.
\end{indented}

% Output
\rSec1[intspct.output]{Output}

\pnum
An application shall output a valid JSON text file that conforms to the
introspection schema (\ref{intspct.schema}) to the file specified
in the options (\ref{intspct.opt.out}).

% Files
\rSec1[intspct.file]{Files}

\pnum
An application can provide an \emph{introspection file} that contains valid
JSON that conforms to the introspection schema
(\ref{intspct.schema}).

\pnum
An \emph{introspection file} shall contain the same information as that produced
from the \grammarterm{std-info-opt} information option
(\ref{intspct.opt.info}).

\pnum
An \emph{introspection file} shall be named the same as the application with
any filename extension replaced with the \verb|stdinfo| filename extension. It
is implementation defined how the filename of the introspection file replaces
the application filename extension with the new \verb|stdinfo| filename
extension.

\begin{note}
For Windows, POSIX, and other platforms replacing the filename extension
	would remove any filename bytes after the last period
	(\ucode{002E} \uname{FULL STOP}) and append the \verb|stdinfo| sequence of
	bytes.
\end{note}

\pnum
An \emph{introspection file} shall either have the same parent directory as
the application, have an implementation defined parent directory that is
relative to the parent directory of the application, or have an implementation
defined parent directory.

% Schema
\rSec1[intspct.schema]{Schema}

\pnum
An introspection JSON text file shall contain one introspection JSON object
(\ref{intspct.schema.obj}).

% Introspection Object
\rSec2[intspct.schema.obj]{Introspection Object}

\pnum
The \emph{introspection object} is the root JSON object of the introspection
JSON text.

\pnum
An \emph{introspection object} can have the following fields.

% JSON Schema Field
\rSec2[intspct.schema.schema]{JSON Schema Field}


\begin{itemdescr}

\pnum \fldname \verb|$schema|

\pnum \fldtype \verb|string|

\pnum \fldval
	The value shall be a reference to a JSON Schema specification.

\pnum \flddesc
	An \emph{introspection object} can contain this field.
	If an \emph{introspection object} does not contain this field the value
	shall be a reference to the JSON Schema corresponding to the current
	edition of this standard.
\end{itemdescr}

% Capability Field
\rSec2[intspct.schema.cap]{Capability Field}


\begin{itemdescr}

\pnum \fldname
	\grammarterm{capability-identifier} (\ref{intspct.cap})

\pnum \fldtype
	\verb|string| or \verb|array|

\pnum \fldval (for \verb|string|)
	The value shall be a \grammarterm{version-number} for \emph{minimum level}
	functionality.
	Or the value shall be a \grammarterm{version-range} for \emph{full level}
	functionality.

\pnum \fldval (for \verb|array|)
	The value can be a JSON \verb|array| for \emph{full level} functionality.
	If the value is a JSON \verb|array| the items in the \verb|array| shall be a
	\grammarterm{version-number} or \grammarterm{version-range}.

\pnum \flddesc
	An \emph{introspection object} can contain this field one or more times.
	When the field appears more than one time the name of the fields shall be
	unique within the \emph{introspection object}.
\end{itemdescr}

% Capabilities
\rSec1[intspct.cap]{Capabilities}

\begin{bnf}

\nontermdef{capability-identifier}\br
name scope-designator name \opt{sub-capability-identifier}

\nontermdef{sub-capability-identifier}\br
scope-designator name \opt{sub-capability-identifier}

\nontermdef{name}\br
\descr{one or more of:}\br
\ucode{0061} .. \ucode{007A} \uname{LATIN SMALL LETTER A .. Z}\br
\ucode{005F} \uname{LOW LINE}

\nontermdef{scope-designator}\br
\ucode{002E} \uname{FULL STOP}
\end{bnf}
\pnum
A \grammarterm{capability-identifier} is composed of two or more
\grammarterm{scope-designator} delimited \grammarterm{name} parts.

\pnum
The \grammarterm{name} \verb|std| in a \grammarterm{capability-identifier} is
reserved for capabilities defined in this standard.

\pnum
Applications can specify vendor designated \grammarterm{name} parts defined
outside of this standard.

% Versions
\rSec1[intspct.vers]{Versions}

\pnum
A version shall be either a single version number (\ref{intspct.vers.num}) or a
version range (\ref{intspct.vers.range}).

\pnum
A single version number shall be equivalent to the inclusive version range
spanning solely that single version number.

\begin{note}
That is, the version number \verb|i.j.k| is equivalent to version range
\verb|[i.j.k,i.j.k]|.
\end{note}

% Version Number
\rSec2[intspct.vers.num]{Version Number}

\pnum
A version number shall conform to the SemVer \verb|<version core>| syntax.

\pnum
A version number can be truncated to only \verb|<major>| or
\verb|<major>.<minor>| syntax.

\pnum
A version number composed of only \verb|<major>| is equivalent to
\verb|<major>.0.0|.

\pnum
A version number composed of only \verb|<major>.<minor>| is equivalent to
\verb|<major>.<minor>.0|.

\pnum
Version numbers define a total ordering where version number \emph{a} is
ordered before a version number \emph{b} when \emph{a} has a lower SemVer
precedence than \emph{b}.

% Version Range
\rSec2[intspct.vers.range]{Version Range}

\begin{bnf}

\nontermdef{version-range}\br
version-range-min-bracket\br
version-min-number \opt{version-range-max-part}\br
version-range-max-bracket

\nontermdef{version-range-max-part}\br
\ucode{002C} \uname{COMMA} version-max-number

\nontermdef{version-min-number}\br
version-number

\nontermdef{version-max-number}\br
version-number

\nontermdef{version-range-min-bracket}\br
\descr{one of:}\br
\ucode{005B} \uname{LEFT SQUARE BRACKET}\br
\ucode{0028} \uname{LEFT PARENTHESIS}

\nontermdef{version-range-max-bracket}\br
\descr{one of:}\br
\ucode{005D} \uname{RIGHT SQUARE BRACKET}\br
\ucode{0029} \uname{RIGHT PARENTHESIS}
\end{bnf}
\pnum
A version range is composed of either one version number bracketed,
or two version numbers separated by a \ucode{002C} \uname{COMMA} and bracketed.

\begin{example}
A version range with a single version number “\verb|[1.0.0]|”.
\end{example}

\begin{example}
A version range with a two version numbers “\verb|[1.0.0,2.0.0]|”.
\end{example}

\pnum
A version range \emph{a} that is \verb|[|\emph{i}\verb|,|\emph{j}\verb|]| makes
\emph{i} and \emph{j} inclusive version range numbers, defining a Mathematics
closed interval.

\pnum
A version range \emph{a} that is \verb|(|\emph{i}\verb|,|\emph{j}\verb|)| makes
\emph{i} and \emph{j} exclusive version range numbers, defining a Mathematics
open interval.

\pnum
A version range \emph{a} that is \verb|(|\emph{i}\verb|,|\emph{j}\verb|]| makes
\emph{i} an exclusive version number and \emph{j} an inclusive version number,
defining a Mathematics half-open interval.

\pnum
A version range \emph{a} that is \verb|[|\emph{i}\verb|,|\emph{j}\verb|)| makes
\emph{j} an exclusive version number.

\pnum
A version range with a single inclusive version number \emph{x} is equivalent
to the version range \verb|[|\emph{x}\verb|,|\emph{x}\verb|]|.

\pnum
A version range with a single exclusive version number \emph{x} is invalid.

\pnum
An exclusive version number \emph{x} does not include the version number
\emph{x} when compared to another version number \emph{y}.

\pnum
A version range \emph{a} with version numbers \emph{i} and \emph{j} when
compared to a version range \emph{b} with version number \emph{m} and \emph{n}
will result in an empty version range when: \emph{j} < \emph{m} or
\emph{n} < \emph{i}.

\pnum
Otherwise if \emph{i} or \emph{m} are inclusive version numbers and if \emph{j}
or \emph{n} are inclusive version numbers the resulting range when \emph{a} is
compare to \emph{b} is the inclusive version numbers “lesser of \emph{i} and
\emph{m}” and “lesser of \emph{j} and \emph{n}”.

\pnum
Otherwise if \emph{i} or \emph{m} are inclusive version numbers and if \emph{j}
or \emph{n} are inclusive version numbers the resulting range when \emph{a} is
compare to \emph{b} is the inclusive version number “lesser of \emph{i} and
\emph{m}” and the exclusive version number “lesser of \emph{j} and \emph{n}”.

\pnum
Otherwise if \emph{j} or \emph{n} are inclusive version numbers the resulting
range when \emph{a} is compared to \emph{b} is the exclusive version number
“lesser of \emph{i} and \emph{m}” and the inclusive version number “lesser of
\emph{j} and \emph{n}”.

\pnum
Otherwise the resulting range when \emph{a} is compared to \emph{b} is the
exclusive version numbers “lesser of \emph{i} and \emph{m}” and “lesser of
\emph{j} and \emph{n}”.

% Minimum Level
\rSec1[intspct.min]{Minimum Level}

\pnum
An application that supports the \emph{minimum level} functionality indicates
it by specifying a single version (\ref{intspct.vers.num}) as the value of the
\verb|std.info| capability (\ref{intspct.cap}).

\begin{example}
\verb|{ "std.info": "1.0.0" }|
\end{example}

% Full Level
\rSec1[intspct.full]{Full Level}

\pnum
An application can support the \emph{full level} functionality as defined in
this section. An application that reports supporting the \emph{full level}
functionality shall support all of the functionality in this section.

\pnum
An application that supports the \emph{full level} functionality indicates it by
specifying a version range (\ref{intspct.vers.range}) or an array of version
range items as the value of the \verb|std.info| capability  (\ref{intspct.cap}).

\begin{example}
\verb|{ "std.info": "[1.0.0]" }|
\end{example}

\pnum
An application that responds with an array of version range items as the value
of a capability field shall support the union of the range items indicated.

% Introspection Information
\rSec1[intspct.info]{Introspection Information}

\pnum
An application shall output an introspection schema (\ref{intspct.schema}) that
contains one capability field for each capability that the application supports
when given the \verb|--std-info| option (\ref{intspct.opt.info}).

\pnum
An application shall indicate the single version (\ref{intspct.vers.num}) or
version range (\ref{intspct.vers.range}) of each capability it supports as the
value of the capability field.

% Introspection Declaration
\rSec1[intspct.dcl]{Introspection Declaration}

\pnum
An application that supports the \emph{full level} functionality when given one
or more \grammarterm{std-info-opt} \grammarterm{declaration} options shall
conform its functionality to the indicated edition of this standard in the given
\grammarterm{declaration} \grammarterm{version-number} for the given capability.

\begin{bnf}

\nontermdef{declaration}\br
capability-identifier \ucode{003d} \uname{equals sign} version-number
\end{bnf}
\pnum
An application, when not given a \grammarterm{std-info-opt}
\grammarterm{declaration} option for a capability it supports, should conform
its functionality to the most recent version of the standard it supports for
that capability.

\pnum
An application, when given a capability declaration option and the given
version is outside of the version range that the application supports, should
indicate an error.

% Compatability
\rSec1[intspct.compat]{Compatability}

\pnum
An application shall indicate, per SemVer specification, that version \emph{n}
of the interface it implements is \emph{backward compatible} with another
version \emph{p} of the interface that another application implements when the
\verb|<major>| number is the same in version \emph{n} and \emph{p} and version
\emph{n} follows version \emph{p}.
