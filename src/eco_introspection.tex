%!TEX root = ecosystem.tex

\rSec0[intspct]{Introspection}

\rSec1[intspct.pre]{Preamble}

\pnum
This clause describes options, output, and formats that describe what
capabilities of this standard an application supports. An application shall
support the \emph{minimum level} functionality \iref{intspct.min}. An
application can support the \emph{full level} functionality \iref{intspct.full}.

\pnum
This clause specifies the \verb|std:info| capability \iref{intspct.cap}.

\rSec1[intspct.overview]{Overview}

\pnum
\begin{outputblock}
@\emph{application}@ [--std-info[=@\emph{declaration}@]] [--std-info-out=@\emph{file}@]
\end{outputblock}

\rSec1[intspct.options]{Options}

\rSec2[intspct.opt.info]{Information Option}

\pnum
This option shall be supported.

\pnum
\verb@--std-info@
\begin{indented}
	Outputs the version information of the capabilities supported by the
	application.
	The option can be specified zero or one time.
	The application shall support the option for \emph{minimum level}
	\iref{intspct.min} functionality.
\end{indented}

\rSec2[intspct.opt.out]{Information Output Option}

\pnum
This option shall be supported.

\pnum
\verb@--std-info-out=@\emph{file}
\begin{indented}
	The pathname of a file to output the information to. If \emph{file} is
	'\verb@-@', the standard output shall be used.
	The application shall support the option for \emph{minimum level}
	\iref{intspct.min} functionality.
	Not specifying this option while specifying the \verb@--std-info@ option
	\iref{intspct.opt.info} shall be equivalent to also specifying a
	\verb@--std-info-out=-@ option.
\end{indented}

\rSec2[intspct.opt.decl]{Declaration Option}

\pnum
This option should be supported.

\pnum
\verb@--std-info=@\emph{declaration}
\begin{indented}
	Declares the required capability version of the application.
	The option can be specified any number of times.
	The application shall support the option for \emph{full level}
	\iref{intspct.full} functionality.
\end{indented}

\rSec1[intspct.output]{Output}

\pnum
An application shall output a valid JSON text file that conforms to the
introspection schema \iref{intspct.schema} to the file specified in the options
\iref{intspct.opt.out}.

\rSec1[intspct.schema]{Schema}

\pnum
An introspection JSON text file shall contain one introspection JSON object
\iref{intspct.schema.obj}.

\rSec2[intspct.schema.obj]{Introspection Object}

\pnum
The \emph{introspection object} is the root JSON object of the introspection
JSON text.

\pnum
An \emph{introspection object} can have the following fields.

\rSec2[intspct.schema.schema]{JSON Schema Field}

\begin{itemdescr}

	\pnum
	\fldname
	\verb|$schema|

	\pnum
	\fldtype
	\verb|string|

	\pnum
	\fldval
	The value shall be a reference to a JSON Schema specification.

	\pnum
	\flddesc
	An \emph{introspection object} can contain this field.
	If an \emph{introspection object} does not contain this field the value
	shall be a reference to the JSON Schema corresponding to the current
	edition of this standard.

\end{itemdescr}

\rSec2[intspct.schema.cap]{Capability Field}

\begin{itemdescr}

	\pnum
	\fldname
	\emph{capability-identifier} \iref{intspct.cap}

	\pnum
	\fldtype
	\verb|string|

	\pnum
	\fldval
	The value shall be a \emph{version-number} for \emph{minimum level}
	functionality.
	Or the value shall be a \emph{version-range} for \emph{full level}
	functionality.
  
	\pnum
	\flddesc
	An \emph{introspection object} can contain this field one or more times.
	When the field appears more than one time the name of the fields shall be
	unique within the \emph{introspection object}.
  
\end{itemdescr}

\rSec1[intspct.cap]{Capabilities}

\pnum
\begin{bnf}

\nontermdef{capability-identifier}\br
	name scope-designator name \opt{sub-capability-identifier}

\nontermdef{sub-capability-identifier}\br
	scope-designator name \opt{sub-capability-identifier}

\nontermdef{name}\br
	\descr{one or more of:}\br
	\terminal{a b c d e f g h i j k l m}\br
	\terminal{n o p q r s t u v w x y z}\br
	\terminal{_}

\nontermdef{scope-designator}\br
	\terminal{:}

\end{bnf}

\pnum
A \emph{capability-identifier} is composed of two or more \emph{scope-designator}
delimited \emph{name} parts.

\pnum
The \emph{name} \verb|std| in a \emph{capability-identifier} is reserved for
capabilities defined in this standard.

\pnum
Applications can specify vendor designated \emph{name} parts defined outside of
this standard.

\rSec1[intspct.vers]{Versions}

\pnum
A version shall be either a single version number \iref{intspct.vers.num} or a
version range \iref{intspct.vers.range}.

\pnum
A single version number shall be equivalent to the inclusive version range
spanning solely that single version number.

\begin{note}
That is, the version number \verb|i.j.k| is equivalent to version range
\verb|[i.j.k,i.j.k]|.
\end{note}

\rSec2[intspct.vers.num]{Version Number}

\pnum
\begin{bnf}

\nontermdef{version-number}\br
	major-number \opt{minor-patch-part}

\nontermdef{minor-patch-part}\br
	\terminal{.} minor-number \opt{patch-part}

\nontermdef{patch-part}\br
	\terminal{.} patch-number

\nontermdef{major-number}\br
	digits

\nontermdef{minor-number}\br
	digits

\nontermdef{patch-number}\br
	digits

\nontermdef{digits}\br
	\descr{one or more of:}\br
	\terminal{0 1 2 3 4 5 6 7 8 9}

\end{bnf}

\pnum
A version number is composed of 1, 2, or 3 decimal numbers (\emph{digits})
separated by a period (\verb|.|).

\pnum
A version number composed of 1 decimal number is equivalent to that decimal
number followed by \verb|.0.0|.

\begin{note}
That is, the version number \verb|N| is equivalent to \verb|N.0.0|.
\end{note}

\pnum
A version number composed of 2 decimal number parts is equivalent to those
decimal number parts followed by \verb|.0|.

\begin{note}
That is, the version number \verb|N.M| is equivalent to \verb|N.M.0|.
\end{note}

\pnum
Version numbers define a total ordering where version number \emph{a} with parts
\emph{i}, \emph{j}, \emph{k} is ordered before version number \emph{b} with
parts \emph{p}, \emph{q}, \emph{r} when: \emph{i} < \emph{p}, or \emph{i} ==
\emph{p} and \emph{j} < \emph{q}, or \emph{i} == \emph{p} and \emph{j} ==
\emph{q} and \emph{k} < \emph{r}.

\pnum
Otherwise version number \emph{a} is ordered before version number \emph{b}
when: \emph{i} > \emph{p}, or \emph{i} == \emph{p} and \emph{j} > \emph{q}, or
\emph{i} == \emph{p} and \emph{j} == \emph{q} and \emph{k} > \emph{r}.

\pnum
Otherwise version number \emph{a} is the same as version number \emph{b}.

\rSec2[intspct.vers.range]{Version Range}

\pnum
\begin{bnf}

\nontermdef{version-range}\br
	version-range-min-bracket version-min-number \opt{version-range-max-part}

\nontermdef{version-range-max-part}\br
	\terminal{,} version-max-number version-range-max-bracket

\nontermdef{version-min-number}\br
	version-number

\nontermdef{version-max-number}\br
	version-number

\nontermdef{version-range-min-bracket}\br
	\descr{one of:}\br
	\terminal{[ (}

\nontermdef{version-range-max-bracket}\br
	\descr{one of:}\br
	\terminal{) ]}

\end{bnf}

\pnum
A version range is composed of either one version number bracketed,
or two version numbers separated by a comma (\verb|,|) and bracketed.

\begin{example}
A version range with a single version number "\verb|[1.0.0]|".
\end{example}

\begin{example}
A version range with a two version numbers "\verb|[1.0.0,2.0.0]|".
\end{example}

\pnum
A version range \emph{a} that is \verb|[|\emph{i}\verb|,|\emph{j}\verb|]| makes
\emph{i} and \emph{j} inclusive version range numbers.

\pnum
A version range \emph{a} that is \verb|(|\emph{i}\verb|,|\emph{j}\verb|)| makes
\emph{i} and \emph{j} exclusive version range numbers.

\pnum
A version range \emph{a} that is \verb|(|\emph{i}\verb|,|\emph{j}\verb|]| makes
\emph{i} an exclusive version number.

\pnum
A version range \emph{a} that is \verb|[|\emph{i}\verb|,|\emph{j}\verb|)| makes
\emph{j} an exclusive version number.

\pnum
A version range with a single inclusive version number \emph{x} is equivalent
to the version range \verb|[|\emph{x}\verb|,|\emph{x}\verb|]|.

\pnum
A version range with a single exclusive version number \emph{x} is invalid.

\pnum
An exclusive version number \emph{x} does not include the version number
\emph{x} when compared to another version number \emph{y}.

\pnum
A version range \emph{a} with version numbers \emph{i} and \emph{j} when
compared to a version range \emph{b} with version number \emph{m} and \emph{n}
will result in an empty version range when: \emph{j} < \emph{m} or
\emph{n} < \emph{i}.

\pnum
Otherwise if \emph{i} or \emph{m} are inclusive version numbers and if \emph{j}
or \emph{n} are inclusive version numbers the resulting range when \emph{a} is
compare to \emph{b} is the inclusive version numbers "lesser of \emph{i} and
\emph{m}" and "lesser of \emph{j} and \emph{n}".

\pnum
Otherwise if \emph{i} or \emph{m} are inclusive version numbers and if \emph{j}
or \emph{n} are inclusive version numbers the resulting range when \emph{a} is
compare to \emph{b} is the inclusive version number "lesser of \emph{i} and
\emph{m}" and the exclusive version number "lesser of \emph{j} and \emph{n}".

\pnum
Otherwise if \emph{j} or \emph{n} are inclusive version numbers the resulting
range when \emph{a} is compared to \emph{b} is the exclusive version number
"lesser of \emph{i} and \emph{m}" and the inclusive version number "lesser of
\emph{j} and \emph{n}".

\pnum
Otherwise the resulting range when \emph{a} is compared to \emph{b} is the
exclusive version numbers "lesser of \emph{i} and \emph{m}" and "lesser of
\emph{j} and \emph{n}".

\rSec1[intspct.min]{Minimum Level}

\pnum
An application that supports the \emph{minimum level} functionality indicates
it by specifying a single version \iref{intspct.vers.single} as the value of the
\verb|std:info| capability \iref{intspct.cap}.

\begin{example}
\verb|{ "std:info": "1.0.0" }|
\end{example}

\rSec1[intspct.full]{Full Level}

\pnum
An application can support the \emph{full level} functionality as defined in
this section. An application that reports supporting the \emph{full level}
functionality shall support all of the functionality in this section.

\pnum
An application that supports the \emph{full level} functionality indicates it by
specifying a version range \iref{intspct.vers.range} as the value of the
\verb|std:info| capability  \iref{intspct.cap}.

\begin{example}
\verb|{ "std:info": "[1.0.0]" }|
\end{example}
	
\rSec1[intspct.info]{Introspection Information}

\pnum
An application shall output an introspection schema \iref{intspct.schema} that
contains one capability field for each capability that the application supports
when given the \verb|--std-info| option \iref{intspct.opt.info}.

\pnum
An application shall indicate the single version \iref{intspct.vers.num} or
version range \iref{intspct.vers.range} of each capability it supports as the
value of the capability field.

\rSec1[intspct.dcl]{Introspection Declaration}

\pnum
An application that supports the \emph{full level} functionality when given one
or more \verb|--std-info=|\emph{declaration} options shall conform its
functionality to the indicated edition of this standard in the given
\emph{declaration} \emph{version-number} for the given capability.

\begin{bnf}

\nontermdef{declaration}\br
	capability-identifier \terminal{=} version-number

\end{bnf}

\pnum
An application, when not given a \verb|--std-info=|\emph{declaration} option for
a capability it supports, should conform its functionality to the most recent
version of the standard it supports for that capability.

\pnum
An application, when given a capability declaration option and the given
version is outside of the version range that the application supports, should
indicate an error.
